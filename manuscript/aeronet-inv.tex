%% Copernicus Publications Manuscript Preparation Template for LaTeX Submissions
%% ---------------------------------
%% This template should be used for copernicus.cls
%% The class file and some style files are bundled in the Copernicus Latex Package, which can be downloaded from the different journal webpages.
%% For further assistance please contact Copernicus Publications at: production@copernicus.org
%% https://publications.copernicus.org/for_authors/manuscript_preparation.html

%% copernicus_rticles_template (flag for rticles template detection - do not remove!)

%% Please use the following documentclass and journal abbreviations for discussion papers and final revised papers.

%% 2-column papers and discussion papers
\documentclass[, manuscript]{copernicus}



%% Journal abbreviations (please use the same for discussion papers and final revised papers)


% Advances in Geosciences (adgeo)
% Advances in Radio Science (ars)
% Advances in Science and Research (asr)
% Advances in Statistical Climatology, Meteorology and Oceanography (ascmo)
% Annales Geophysicae (angeo)
% Archives Animal Breeding (aab)
% ASTRA Proceedings (ap)
% Atmospheric Chemistry and Physics (acp)
% Atmospheric Measurement Techniques (amt)
% Biogeosciences (bg)
% Climate of the Past (cp)
% DEUQUA Special Publications (deuquasp)
% Drinking Water Engineering and Science (dwes)
% Earth Surface Dynamics (esurf)
% Earth System Dynamics (esd)
% Earth System Science Data (essd)
% E&G Quaternary Science Journal (egqsj)
% Fossil Record (fr)
% Geochronology (gchron)
% Geographica Helvetica (gh)
% Geoscience Communication (gc)
% Geoscientific Instrumentation, Methods and Data Systems (gi)
% Geoscientific Model Development (gmd)
% History of Geo- and Space Sciences (hgss)
% Hydrology and Earth System Sciences (hess)
% Journal of Micropalaeontology (jm)
% Journal of Sensors and Sensor Systems (jsss)
% Mechanical Sciences (ms)
% Natural Hazards and Earth System Sciences (nhess)
% Nonlinear Processes in Geophysics (npg)
% Ocean Science (os)
% Primate Biology (pb)
% Proceedings of the International Association of Hydrological Sciences (piahs)
% Scientific Drilling (sd)
% SOIL (soil)
% Solid Earth (se)
% The Cryosphere (tc)
% Web Ecology (we)
% Wind Energy Science (wes)


%% \usepackage commands included in the copernicus.cls:
%\usepackage[german, english]{babel}
%\usepackage{tabularx}
%\usepackage{cancel}
%\usepackage{multirow}
%\usepackage{supertabular}
%\usepackage{algorithmic}
%\usepackage{algorithm}
%\usepackage{amsthm}
%\usepackage{float}
%\usepackage{subfig}
%\usepackage{rotating}


% The "Technical instructions for LaTex" by Copernicus require _not_ to insert any additional packages.
%
\usepackage{algorithmic}
\usepackage{algorithm}


\begin{document}

\title{Associations between optical, physical and chemical properties of
aerosols measured at ground-based networks}


\Author[1]{Meredith}{Franklin}
\Author[2]{Meytar}{Sorek-Hamer}
\Author[3]{Olga}{Kalashnikova}
\Author[3]{Michael}{Garay}
\Author[3]{David}{Diner}


\affil[1]{University of Southern California, Los Angeles, CA, United States}
\affil[2]{NASA Ames Research Center, Moffett Field, CA, USA}
\affil[3]{Jet Propulsion Laboratory, Pasadena, CA, USA}

%% The [] brackets identify the author with the corresponding affiliation. 1, 2, 3, etc. should be inserted.



\runningtitle{Physical and optical properties of aerosols}

\runningauthor{Franklin et al.}


\correspondence{Meredith\ Franklin\ (meredith.franklin@usc.edu)}



\received{}
\pubdiscuss{} %% only important for two-stage journals
\revised{}
\accepted{}
\published{}

%% These dates will be inserted by Copernicus Publications during the typesetting process.


\firstpage{1}

\maketitle


\begin{abstract}
The abstract goes here. It can also be on \emph{multiple lines}.
\end{abstract}


\copyrightstatement{The author's copyright for this publication is transferred to
institution/company.}


\introduction

A large body of literature has shown that satellite-derived aerosol
optical depth (AOD), typically retrieved at 550 nm wavelength, reliably
correlates with mass-volume concentrations of fine mode particulate
matter with aerodynamic diameter less than 2.5 \(\mu\)m (PM\(_{2.5}\)).
Studies that have used satellite observations to generate PM\(_{2.5}\)
have been instrumental for air pollution - health effects research. The
associations between AOD and different chemical components of
PM\(_{2.5}\) are lesser known. A handful of studies using the Multiangle
Imaging SpectroRadiometer (MISR), an instrument onboard the NASA Terra
satellite that provides observations of optical properties by particle
type (size, shape, absorption), have provided evidence that different
optical properties relate to different physical and chemical properties
of particulate matter. Results have been somewhat inconsistent, showing
differences depending on geographic area of analysis, optical components
used, and statistical tools applied. The purpose of this analysis is to
make a detailed examination of the statistical relationships between
ground-level PM2.5 and PM2.5 chemical components (nitrate, sulfate,
elemental carbon, organic carbon, dust) and optical measures of aerosols
(e.g.~aerosol optical depth, angstrom exponent).

AERONET \cite{Holben1998, Shin2018, Shin2018b, Shin2019}

\begin{itemize}
\item
  Specific sections of the manuscript:

  \begin{itemize}
  \item
    \texttt{running} with \texttt{title} and \texttt{author}
  \item
    \texttt{competinginterests}
  \item
    \texttt{copyrightstatement} (optional)
  \item
    \texttt{availability} (strongly recommended if any used), one of
    \texttt{code}, \texttt{data}, or \texttt{codedata}
  \item
    \texttt{authorcontribution}
  \item
    \texttt{disclaimer}
  \item
    \texttt{acknowledgements}
  \end{itemize}
\end{itemize}

See the defaults and examples from the skeleton and the official
Copernicus documentation for details.

\textbf{Important}: Always double-check with the official manuscript
preparation guidelines at
\url{https://publications.copernicus.org/for_authors/manuscript_preparation.html},
especially the sections ``Technical instructions for LaTeX'' and
``Manuscript composition''. Please contact Daniel Nüst,
\texttt{daniel.nuest@uni-muenster.de}, with any problems.

\section{Methods}

The study encompasses the San Joaquin Valley region of central
California (Figure 1). We leverage datasets available at four sites -
Bakersfield, Fresno, Modesto, and Visalia. At these sites there are
co-located instruments from EPA's chemical speciation network (CSN),
EPA's air quality system (AQS) and NASA's AERONET network.

\subsection{Data}

The CSN monitors are on a 1-in-3 or 1-in-6 day sampling schedule,
providing PM\(_{2.5}\)mass and component PM\(_{2.5}\) concentrations of
metals (e.g.~Aluminium Al, Silicon Si, Calcium Ca, Titanium Ti, Iron Fe)
obtained from X-ray fluorescence (XRF), ions (nitrate NO\(_{3}^-\) and
sulfate SO\(_4^{2-}\)) from ion chromatography, and carbons (organic OC
and elemental EC) from thermal/optical analysis. To quantify dust we use
the following equation \citet{Chow2015}: dust = 2.2Al + 2.49Si + 1.63Ca
+ 1.94Ti + 2.42Fe

The AQS monitors provide daily concentrations of PM\(_{2.5}\) mass by
the EPA's Federal Reference Method, which is the highest quality
gravimetric measurement method used for regulatory purposes.

AERONET sites are sunphotometers providing a ``ground-up'' measurement
of aerosol optical properties at multiple wavelengths and have been used
extensively to validate ``top-down'' satellite observations of related
properties. Wavelength-specific AOD and angstrom exponents are the
primary sunphotometer variables. Using quadratic log-log interpolation
we calculated AOD 550 nm from AOD 440, 500, 675, 870 nm in log-log
space. AOD at 550 nm is the most common wavelength retrieved from
satellite instruments. A retrieval-based AERONET product, called the
inversion product, provides an additional suite of aerosol properties
that help distinguish size (fine, coarse effective radius), shape
(asymmetry), and absorption. We excluded sunphotometer and inversion
variables that had a significant proportion of missing data
(\textasciitilde{}90\% missing). A list of the variables included in the
analysis are shown in the Appendix. In a separate test we examine data
from the SPARTAN site in Rehovot, Israel. The SPARTAN network provides
data for PM\(_{2.5}\) mass and speciation concentrations on an
integrated 1 in 9 day sampling schedule, and is colocated with an
AERONET site (We don't have the speciation data for this site so we
could only look at PM\(_{2.5}\) for now).

\subsection{Statistical methods}

Prior to model building we examined a cluster-based correlation heat map
(Figure 2), which provides the Pearson correlations between all pairs of
AERONET variables grouped by a decision tree. To avoid collinearity in
the regression models, we kept the most relevant of a group of variables
that had a correlation coefficient \textgreater{}
\textbar{}0.9\textbar{}. We then examined and picked a subset of
variables connected at the mid-tier level of the tree to construct
interactions. We fit simple linear regression models separately for
PM2.5 mass, sulfate, nitrate, EC, OC, and dust with AOD 550 nm as the
sole predictor variable. Multiple linear regression models were again
fit separately for PM2.5 mass, sulfate, nitrate, EC, OC, and dust, but
with the combined AERONET sunphotometer and inversion product as
predictor variables and model selection was conducted using the ``all
possible subset method''. This method constructs models based on all
combinations from 1 to k variable models. We select the best model from
the combinations based on highest R2, lowest RMSE, and Mallow's Cp
statistic that is close to k+1. Model selection for the Fresno and
Bakersfield sites were examined separately and in combination in a
``total CA'' analysis, which combined data from Fresno, Bakersfield,
Modesto, Visalia, and a special DRAGON campaign in late 2012-early 2013
over the region (8 co-located sites with PM2.5 mass).

All models were cross validated (CV) with 10-fold CV, and we report the
CV R2 and RMSE. Models were fit in R using the leaps() library.

\section{Results}

\section{Content section with R code chunks}

You should always use \texttt{echo\ =\ FALSE} on R Markdown code blocks
as they add formatting and styling not desired by Copernicus. The hidden
workflow results in 42.

You can add verbatim code snippets without extra styles by using
\texttt{\textasciigrave{}\textasciigrave{}\textasciigrave{}} without
additional instructions.

\begin{verbatim}
sum <- 1 + 41
\end{verbatim}

\section{Content section with list}

If you want to insert a list, you must

\begin{itemize}
\item
  leave
\item
  empty lines
\item
  between each list item
\end{itemize}

because the \texttt{\textbackslash{}tightlist} format used by R Markdown
is not supported in the Copernicus template. Example:

\begin{verbatim}
- leave

- empty lines

- between each list item
\end{verbatim}

\section{Examples from the official template}

\subsection{FIGURES}

When figures and tables are placed at the end of the MS (article in
one-column style), please add \clearpage between bibliography and first
table and/or figure as well as between each table and/or figure.

\subsubsection{ONE-COLUMN FIGURES}

Include a 12cm width figure of Nikolaus Copernicus from
\href{https://en.wikipedia.org/wiki/File:Nikolaus_Kopernikus.jpg}{Wikipedia}
with caption using R Markdown.

\begin{figure}
\includegraphics[width=8.3cm]{Nikolaus_Kopernikus} \caption{one column figure}\label{fig:unnamed-chunk-2}
\end{figure}

\subsubsection{TWO-COLUMN FIGURES}

You can also include a larger figure.

\begin{figure}
\includegraphics[width=12cm]{Nikolaus_Kopernikus} \caption{two column figure}\label{fig:unnamed-chunk-3}
\end{figure}

\subsection{TABLES}

You can ad \LaTeX table in an R Markdown document to meet the template
requirements.

\subsubsection{ONE-COLUMN TABLE}

\begin{table}[t]
\caption{TEXT}
\begin{tabular}{l c r}
\tophline

a & b & c \\
\middlehline
1 & 2 & 3 \\

\bottomhline
\end{tabular}
\belowtable{Table Footnotes}
\end{table}

\subsubsection{TWO-COLUMN TABLE}

\begin{table*}[t]
\caption{TEXT}
\begin{tabular}{l c r}
\tophline

a & b & c \\
\middlehline
1 & 2 & 3 \\

\bottomhline
\end{tabular}
\belowtable{Table footnotes}
\end{table*}

\subsection{MATHEMATICAL EXPRESSIONS}

All papers typeset by Copernicus Publications follow the math
typesetting regulations given by the IUPAC Green Book (IUPAC:
Quantities, Units and Symbols in Physical Chemistry, 2nd Edn., Blackwell
Science, available at:
http://old.iupac.org/publications/books/gbook/green\_book\_2ed.pdf,
1993).

Physical quantities/variables are typeset in italic font (t for time, T
for Temperature)

Indices which are not defined are typeset in italic font (x, y, z, a, b,
c)

Items/objects which are defined are typeset in roman font (Car A, Car B)

Descriptions/specifications which are defined by itself are typeset in
roman font (abs, rel, ref, tot, net, ice)

Abbreviations from 2 letters are typeset in roman font (RH, LAI)

Vectors are identified in bold italic font using \vec{x}

Matrices are identified in bold roman font

Multiplication signs are typeset using the LaTeX commands
\texttt{\textbackslash{}times} (for vector products, grids, and
exponential notations) or \texttt{\textbackslash{}cdot}

The character * should not be applied as mutliplication sign

\subsection{EQUATIONS}

\subsubsection{Single-row equation}

Unnumbered equations (i.e.~using \texttt{\$\$} and getting inline
preview in RStudio) are not supported by Copernicus.

\begin{equation}
1 \times 1 \cdot 1 = 42
\end{equation}

\begin{equation}
A = \pi r^2
\end{equation}

\begin{equation}
x=\frac{2b\pm\sqrt{b^{2}-4ac}}{2c}.  
\end{equation}

\subsubsection{Multiline equation}

\begin{align}
& 3 + 5 = 8\\
& 3 + 5 = 8\\
& 3 + 5 = 8
\end{align}

\subsection{MATRICES}

\[
\begin{matrix}
x & y & z\\
x & y & z\\
x & y & z\\
\end{matrix}
\]

\subsection{ALGORITHM}

If you want to use algorithms, you can either enable the required
packages in the header (the default, see \texttt{algorithms:\ true}), or
make sure yourself that the \LaTeX packages \texttt{algorithms} and
\texttt{algorithmicx} are installed so that \texttt{algorithm.sty}
respectively \texttt{algorithmic.sty} can be loaded by the Copernicus
template. Copernicus staff will remove all undesirable packages from
your LaTeX source code, so please stick to using the header option,
which only adds the two acceptable packages.

\begin{algorithm}
\caption{Algorithm Caption}
\label{a1}
\begin{algorithmic}
\STATE $i\gets 10$
\IF {$i\geq 5$} 
        \STATE $i\gets i-1$
\ELSE
        \IF {$i\leq 3$}
                \STATE $i\gets i+2$
        \ENDIF
\ENDIF
\end{algorithmic}
\end{algorithm}

\subsection{CHEMICAL FORMULAS AND REACTIONS}

For formulas embedded in the text, please use
\texttt{\textbackslash{}chem\{\}}, e.g. \chem{A \rightarrow B}.

The reaction environment creates labels including the letter R, i.e.
(R1), (R2), etc.

\begin{itemize}
\item
  \texttt{\textbackslash{}rightarrow} should be used for normal
  (one-way) chemical reactions
\item
  \texttt{\textbackslash{}rightleftharpoons} should be used for
  equilibria
\item
  \texttt{\textbackslash{}leftrightarrow} should be used for resonance
  structures
\end{itemize}

\begin{reaction}
A \rightarrow B \\
\end{reaction}
\begin{reaction}
Coper \rightleftharpoons nicus \\
\end{reaction}
\begin{reaction}
Publi \leftrightarrow cations
\end{reaction}

\subsection{PHYSICAL UNITS}

Please use \texttt{\textbackslash{}unit\{\}} (allows to save the
math/\texttt{\$} environment) and apply the exponential notation, for
example \(3.14\,\unit{km\,h^{-1}}\) (using LaTeX mode:
\texttt{\textbackslash{}(\ 3.14\textbackslash{},\textbackslash{}unit\{...\}\ \textbackslash{})})
or \unit{0.872\,m\,s^{-1}} (using only
\texttt{\textbackslash{}unit\{0.872\textbackslash{},m\textbackslash{},s\^{}\{-1\}\}}).

\conclusions

The conclusion goes here. You can modify the section name with
\texttt{\textbackslash{}conclusions{[}modified\ heading\ if\ necessary{]}}.




\codedataavailability{use this to add a statement when having data sets and software code
available} %% use this section when having data sets and software code available

\sampleavailability{use this section when having geoscientific samples available} %% use this section when having geoscientific samples available


%%%%%%%%%%%%%%%%%%%%%%%%%%%%%%%%%%%%%%%%%%
%% optional

%%%%%%%%%%%%%%%%%%%%%%%%%%%%%%%%%%%%%%%%%%
\appendix
\section{Figures and tables in appendices}

Regarding figures and tables in appendices, the following two options
are possible depending on your general handling of figures and tables in
the manuscript environment:

\subsection{Option 1}

If you sorted all figures and tables into the sections of the text,
please also sort the appendix figures and appendix tables into the
respective appendix sections. They will be correctly named
automatically.

\subsection{Option 2}

If you put all figures after the reference list, please insert appendix
tables and figures after the normal tables and figures.

To rename them correctly to A1, A2, etc., please add the following
commands in front of them: \texttt{\textbackslash{}appendixfigures}
needs to be added in front of appendix figures
\texttt{\textbackslash{}appendixtables} needs to be added in front of
appendix tables

Please add \texttt{\textbackslash{}clearpage} between each table and/or
figure. Further guidelines on figures and tables can be found below.
\noappendix

%%%%%%%%%%%%%%%%%%%%%%%%%%%%%%%%%%%%%%%%%%
\authorcontribution{M. Franklin conducted analyses and wrote the manuscript. M. Sorek-Hamer
conducted analyses and reviewed the manuscript. O. Kalashnikova and D.
Diner conceptualized the study. D. Diner edited the manuscript.} %% optional section

%%%%%%%%%%%%%%%%%%%%%%%%%%%%%%%%%%%%%%%%%%
\competinginterests{The authors declare no competing interests.} %% this section is mandatory even if you declare that no competing interests are present

%%%%%%%%%%%%%%%%%%%%%%%%%%%%%%%%%%%%%%%%%%
\disclaimer{We like Copernicus.} %% optional section

%%%%%%%%%%%%%%%%%%%%%%%%%%%%%%%%%%%%%%%%%%
\begin{acknowledgements}
This work was supported by NASA Grant 80NSSC19K0225
\end{acknowledgements}

%% REFERENCES
%% DN: pre-configured to BibTeX for rticles

%% The reference list is compiled as follows:
%%
%% \begin{thebibliography}{}
%%
%% \bibitem[AUTHOR(YEAR)]{LABEL1}
%% REFERENCE 1
%%
%% \bibitem[AUTHOR(YEAR)]{LABEL2}
%% REFERENCE 2
%%
%% \end{thebibliography}

%% Since the Copernicus LaTeX package includes the BibTeX style file copernicus.bst,
%% authors experienced with BibTeX only have to include the following two lines:
%%
\bibliographystyle{copernicus}
\bibliography{aeronet.bib}
%%
%% URLs and DOIs can be entered in your BibTeX file as:
%%
%% URL = {http://www.xyz.org/~jones/idx_g.htm}
%% DOI = {10.5194/xyz}


%% LITERATURE CITATIONS
%%
%% command                        & example result
%% \citet{jones90}|               & Jones et al. (1990)
%% \citep{jones90}|               & (Jones et al., 1990)
%% \citep{jones90,jones93}|       & (Jones et al., 1990, 1993)
%% \citep[p.~32]{jones90}|        & (Jones et al., 1990, p.~32)
%% \citep[e.g.,][]{jones90}|      & (e.g., Jones et al., 1990)
%% \citep[e.g.,][p.~32]{jones90}| & (e.g., Jones et al., 1990, p.~32)
%% \citeauthor{jones90}|          & Jones et al.
%% \citeyear{jones90}|            & 1990

\end{document}
